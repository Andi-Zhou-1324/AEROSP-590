\documentclass[a4paper]{article}
%% Language and font encodings
\usepackage[english]{babel}
\usepackage[utf8x]{inputenc}
\usepackage[T1]{fontenc}
\usepackage{comment}


%% Sets page size and margins
\usepackage[a4paper,top=3cm,bottom=2cm,left=3cm,right=3cm,marginparwidth=1.75cm]{geometry}

%% Useful packages
\usepackage{amsmath}
\usepackage{amssymb}
\usepackage{graphicx}

\usepackage{algorithm} 
\usepackage{algpseudocode} 


% For multiple pages in a pdf document
\usepackage{pdfpages}

\usepackage{listings}
\usepackage{color} %red, green, blue, yellow, cyan, magenta, black, white
\definecolor{mygreen}{RGB}{28,172,0} % color values Red, Green, Blue
\definecolor{mylilas}{RGB}{170,55,241}

\lstset{language=Matlab,%
    %basicstyle=\color{red},
    breaklines=true,%
    morekeywords={matlab2tikz},
    keywordstyle=\color{blue},%
    morekeywords=[2]{1}, keywordstyle=[2]{\color{black}},
    identifierstyle=\color{black},%
    stringstyle=\color{mylilas},
    commentstyle=\color{mygreen},%
    showstringspaces=false,%without this there will be a symbol in the places where there is a space
    numbers=left,%
    numberstyle={\tiny \color{black}},% size of the numbers
    numbersep=9pt, % this defines how far the numbers are from the text
    emph=[1]{for,end,break},emphstyle=[1]\color{red}, %some words to emphasise
    %emph=[2]{word1,word2}, emphstyle=[2]{style},    
}
\title{AEROSP 590: Entropy Condition}
\author{Andi Zhou}
\begin{document}
\maketitle

\section{Entropy Condition - Origin and Formulation}
In this section we explain the entropy condition, and explain why this would be necessary in solving Hyperbolic Partial Differential Equations. To explain the neccesity of entropy condition, we first start but discussing a \textit{single conservation law}.

\begin{equation} \label{singleConservationLaw}
    u_t + f_x = 0
\end{equation}
where $f$ is a non-linear function of $U$, where:
\begin{equation}
    \frac{df}{du} = a(u)
\end{equation}
Using chain rule, we could write Equation \ref{singleConservationLaw} as:
\begin{equation}
    u_t + a(u)u_x = 0
\end{equation}
If $u$ is constant along a characteristic trajectory, we also know that $a$ will be constant as well. $u$ would propagate along space in speed $a$. Therefore, we also define $a$ as the signal speed. The trajectories of $u$ would be defined as a characteristics. Equation \ref{singleConservationLaw} is known as the strong (differential) form of the PDE, where a solution to this equation would need to satisfy the differential operators at all points in space and time. A solution that matches the strong form is known as the a strong solution. However, the differential form could not admit discontinuities. We now consider the weak form:

\begin{equation}
    \int_{0}^{\infty} \int_{-\infty}^{\infty} \left[w_t u + w_x f(u) \right] dx dt + \int_{\infty}^\infty w(x,0) \phi(x) dx = 0
\end{equation}
where $w$ is any smooth test function with compact support (non-zero only within the specified element). The general idea of weak solution is that we could rewrite the derivatives of $u$ as derivatives of compact test function $w$. From this, we basically eliminated the need for solution to be continuous. However, for hyperbolic systems (which we are interested in), the notion of weak solution \textbf{does not} guarantee uniqueness. Therefore, we need some sort of criterions to select a solution that is \textit{physical}. We then have the \textbf{entropy} condition. 

It is rather difficult to trace when the notion of entropy condition was first introduced. Earliest written work that could be traced on this subjects were done by P. Lax [citation] and O. Oleinik [citation]. Both authors have mentioned the idea of \textit{entropy conditions} repeatedly in their text. Oleinik [citation needed] has shown that physically relevant solutions must have the following properties:
\begin{equation}
    \frac{f(u) - f(u_L)}{u-u_L} \geq S \geq \frac{f(u) - f(u_R)}{u-u_R}
\end{equation}
This is known by Oleinik as \textit{Condition E}. If we use the Rankine-Hugoniot jump relation:
\begin{equation}
    f(u_R) - f(u_L) = S(u_R - u_L)
\end{equation}
we also arrive at the \textit{Lax entropy condition}:
\begin{equation}
    a(u_l) > S > a(u_r)
\end{equation}
$S$ in the above equation is the speed of discontinuity. In essence, the entropy condition defines that characteristic line should ALWAYS run into the shock, not away from it. 

\section{Entropy Satisfied Fluxes}
Several fluxes satisfy the above entropy condition. We understood from Lax [] that any monotone scheme satisfy the entropy condition. In this section we specifically focus on first order scheme, and list two of these fluxes below.

\subsection{Godunov's Scheme}
MORE TO BE ADDED
\subsection{Engquist-Osher Scheme}
The E-O scheme, or Engquist and Osher scheme could be considered as a modified version of the Cole-Murman scheme used for small disturbance equation of transonic flow.  Engquist and Osher introduced this scheme in a paper publised in 1980 [] and then later generalized this scheme for general hyperbolic systems of conservation laws []. We discuss the formulation of this particular scheme here along with its relevant mathematical proofs.
\subsubsection{General E-O Algorithm}
Consider a non-linear scalar conservation law in 1D:
\begin{equation} 
    w_t + f(w)_x = 0,\\ w(x,t=0) = \Phi(x). \ =\infty < x< \infty
\end{equation}
The solution is approximated by a mesh function $w_j^n$ on the mesh ${(x_J,t^n)}$ with $x_j = k \Delta x, t^n = n \Delta t, j = 0, \pm 1,...,n=0,1,...$. In explicit form, the finite difference approximation is defined as:
\begin{enumerate}
    \item 
\begin{equation} \label{eq:E-O Conservation Law}
    \begin{split}
        w_j^{n+1} &= w_j^n - \frac{\Delta t}{\Delta x}\left(\Delta_{+} f_{-} (w_j^n) + \Delta_{-} f_{+} (w_j^n) \right) \\
        w_j^0 &= \Phi(x_j), j = 0, \pm 1, \pm 2\\
    \end{split}
\end{equation}
    \item
        \begin{equation}
            \frac{\Delta t}{\Delta x} \text{sup} |f'| < 1
        \end{equation}

\end{enumerate}
We denote here as a reminder that Part 1 is the fully-discrete version of Equation \ref{eq:E-O Conservation Law} written is a per cell. We specify here as well the following notations:
\begin{equation}
    \begin{split}
        \Delta_{\pm} w_j &= \pm (w_{j\pm 1} - w_j)\\
        D_{\pm} w_j &= \frac{1}{\Delta x} \Delta_{\pm} w_j\\
    \end{split} 
\end{equation}
The auxiliary functions $f_+$ and $f_-$ are:
\begin{equation}
    \begin{split}
        f_+ (w) &= \int_0^w \mathcal{X} (w) f'(w) dw\\
        f_- (w) &= \int_0^w (1 - \mathcal{X}(w)) f'(w)dw\\
    \end{split}
\end{equation}
When $f$ is convex, the definitions then reduce to:
\begin{equation}
    \begin{split}
        f_+ (w) &= f(\mathrm{max}(w,\bar{w}))\\
        f_- (w) &= f(\mathrm{min}(w,\bar{w})) \\
    \end{split}
\end{equation}

where $\bar{u}$ is the stagnation or sonic point for which $f'(\bar{u}) = 0$. We make a note that when the sign is positive, Equation \ref{eq:E-O Conservation Law} becomes the \textit{upwind} algorithm. This is also known as \textit{flux decomposition}, where we separated the original upwinding flux into essentially a decreasing and increasing components. Decomposing this flux into different components provides us with a different perspective when looking at states near shock points.

When we consider a hyperbolic PDE with convex flux, we arrive at the following:
\begin{equation}
    \begin{split}
        f^+ (w) &= 
        \begin{cases}
            f(w), & w \geq \bar{w}\\
            f(\bar{w}), & w \leq \bar{w}\\
        \end{cases}\\
        f^-(w) &= 
        \begin{cases}
            f(w), & w \leq \bar{w} \\
            f(\bar{w}), & w \geq \bar{w}\\
        \end{cases}\\
    \end{split}
\end{equation}
If we have Burger's equation, then the flux becomes: $f(u) = \frac{u^2}{2}$:

\begin{equation}
    \begin{split}
        f^+ (u) &= 
        \begin{cases}
            \frac{u^2}{2}, & u \geq u^*\\
            0, & u \leq u^*\\
        \end{cases}\\
        f^-(u) &= 
        \begin{cases}
            \frac{u^2}{2}, & u \leq u^* \\
            0, & u \geq u^*\\
        \end{cases}\\
    \end{split}
\end{equation}
If we consider an arbitary, following the notation from Equation \ref{eq:E-O Conservation Law}, we have, 
\begin{equation}
    \Delta_- f_+ = f^+(w_j) + f^-({w_{j+1}})
\end{equation}
The Engquist and Osher scheme demonstrated computational advantages over all previous schemes including Godunov's and Cole-Murman, especially for implicit calculations where a much large time steps could be taken for E-O. It also elimnates the non-physical expansion shocks (satisfies the entropy condition) that plagues the C-M and Roe's method, and also possesses \textit{smoother} flux functions than that of Godunov's.

\subsubsection{Details on Integration Path}
TO BE ADDED
\subsubsection{Proof on E-O Satisfies the Entropy Condition}
TO BE ADDED

\section{The Roe Scheme}




\end{document}